\documentclass[a4paper]{article}

\usepackage[utf8]{inputenc}
\usepackage[T1]{fontenc}
\usepackage{textcomp}
\usepackage{amsmath, amssymb}


% figure support
\usepackage{import}
\usepackage{xifthen}
\pdfminorversion=7
\usepackage{pdfpages}
\usepackage{transparent}
\newcommand{\incfig}[1]{%
	\def\svgwidth{\columnwidth}
	\import{./figures/}{#1.pdf_tex}
}

\pdfsuppresswarningpagegroup=1

\usepackage{fancyhdr}
	\fancyhead{}  		% Clears all page headers and footers
	\pagestyle{fancy}
	\lhead{MTHE 332} 	% TODO: Add automatic context switching, auto complete current class
	\rhead{Joe Grosso}
	\chead{Lesson 1}	% TODO: Smart class auto complete
	\cfoot{ 2020-01-06 }

\begin{document}
	\begin{itemize}
		\item No required textbook
		\item Supplemental lecture notes: a list of additional references have been notes.
		\item No tutorial week 1!		
	\end{itemize}	
	
	\section{Introduction}
	In MTHE 237, 
	\[
		\frac{dx}{dt} = f(x)
	.\] 
	Equations like this have explicit solution (existence/uniqueness implies an explicit solution).

	\begin{itemize}
		\item stability of solutions \implies $\lim_{n \to \infty} |x(t)|$
	\end{itemize}

	In many applications, we also have the liberty of including a variable, called a control variable, to shape the behavior of the equation/system. 
	\[
		\frac{dx}{dt} = f(x,u)
	.\] 	
	In this case $u$ is our control variable. 

	Control Theory studies the problem of selecting a "$u$" subject to design, information and objective constraints. 

	Simple example: How do we control a circuit? 
	
	% Insert diagram here % 
	\[
	i(t) = {\frac{dQ}{dt}}=C \frac{dV_{C}}{dt}\\

	Q = CV_{C}, \\
	
	Ri(t) + V_{C}(t) = e(t)\\


	RC \frac{dV_{C}(t)}{dt} + V_{C}(t) = e(t)\\

	\frac{dV_{C}(t)}{dt} = -\frac{1}{RC} V_{C}(t) + \frac{e(t)}{RC}\\
	
	.\] 
	
	\end{align}

	Exercise:

	\[
		\frac{d}{dt}e^{}
	.\] 

	In the discussion so far, the input, $u(t)$ or $e(t)$, did not depend on the past behavior/readings of the system. That is, $u$ (or $e$ ) was external/open loop. 

	
\end{document}


