\lecture{12}{Fri 31 Jan 2020 14:32}{Convolutions}

Last time: A system for equations:
\begin{align*}
	\left( *  \right) \quad	\theta ^{\left( 1 \right) } &=  A\left( \theta \right) \beta \\
	\theta &\in D'\left( \R; X \right)   \\
	\beta &\in D'\left( \R; X \right) \\
	D'\left( \R; X \right) &= \left\{ \text{continuous linear mappings from }D\left( \R; \R \right) \to X \right\} 
.\end{align*}
where $x_0 \in  X$, $\theta \in  D'\left( \R; \R \right) $:
\begin{align*}
	x_0 \otimes \theta &\in  D'\left( \R; X \right) \\
	\left( x_0 \otimes \theta  \right) \left( \phi \right)  &= \theta \left( \phi \right) x_0
.\end{align*}
if $\theta \in  D'\left( \R; X \right) $ and $A \in  L\left( X; X \right) $, $A\left( \theta  \right) \in  D'\left( \R; X \right) $ 
\[
	\left( A\left( \theta  \right)  \right) \left( \phi \right)  = A\left( \theta \left( \phi \right)  \right) 
.\] 

\begin{notation}
	If $\xi \in  C^{\infty}\left( \R; X \right) $ then we can choose a basis $\left\{ e_{1} , \ldots , e_{n} \right\} $ for $X$ and write 
	\[
		\xi\left( t \right)  = \xi_{1}\left( t \right) e_1 + \ldots + \xi_{n}\left( t \right) e_{n}
	.\] 
	for $\xi_{1} , \ldots , \xi_{n} \in  C^{\infty}\left( \R; \R \right) $. If $\theta \in  D'\left( \R; \R \right) $, define 
	\[
		\xi \otimes \theta \in  D'\left( \R; X \right) 
	.\] by
	\[
		\left( \xi \otimes \theta  \right) \left( \phi \right)  = \left( \xi _{1} \theta  \right) \left( \phi \right)  + \ldots + \left( \xi _{n}\theta  \right) \left( \phi \right) 
	.\] 
	Then, if $\xi\left( t \right)  = e^{At}x_0$ then $\theta _{1 _{\ge 0}\xi}$ is the unique solution in $D'_{+}\left( \R; X \right) $ to $\left( * \right) $ with $\beta = x_0 \otimes \delta$. Indeed, 
	\begin{align*}
		\theta _{1_{\ge 0 }\xi} &= \xi \otimes \theta _{1 _{\ge 0}} \\
		\left( \xi \otimes \theta _{1 _{\ge  0}} \right) ^{\left( 1 \right) } &=  \xi ^{\left( 1 \right) } \otimes \theta _{1 _{\ge  0}} + \xi \otimes \theta _{1 _{\ge  0}}^{\left( 1 \right) } \quad\quad  \left(  \theta _{1 _{\ge  0}}^{\left( 1 \right) } = \delta\right) \\
%		\left\{ \xi \otimes \delta &= \xi\left( 0 \right)  \otimes \delta  \right\} 
		&= A\xi \otimes \theta _{1 _{\ge 0}} + x_0 \otimes \delta  \\
		&= A\left( \xi \otimes \theta _{1 _{\ge  0}} \right)  + x_0 \otimes \delta
	.\end{align*}
	\begin{punch}
	One can use distributions with systems of differential equations, but you need notation to set this up.
	\end{punch}
\end{notation}

Thus endeth distributions!

\section{Convolution}

\begin{table}[htpb]
	\centering
	\caption{Discrete vs Continuous Time Convolutions}
	\label{tab:convolutions}
	\begin{tabular}{ || c c || }
		\hline
	 Continuous Time & Discrete Time \\
	 \hline \hline\\ 
	 $f * g\left( t \right)  = \int_{\R}^{ } f \left( t - s  \right) g\left( s \right) ds $  & $f * g \left( k \Delta \right) = \sum_{j = -\infty}^{\infty} f \left( \left( k - j  \right) \Delta \right) g \left( j\Delta \right) $\\
	 \\
	 \hline
	\end{tabular}
\end{table}

Both of these are convolutions of $f $ and $g$. We will start with the continuous case first. 

Denote by $D \left( f, g \right)  = \left\{ t \in  \R  \mid  s \longmapsto f \left( t - s  \right) g \left( s \right)  \text{ is in }L^{1}\left( \R; \F \right)  \right\} $. 

Say that $\left( f, g \right)   $ is convolvable if $R \setminus D \left( f, g \right) $ has measure zero. Exact conditions under which $\left( f, g \right)  $ is convolvable are not really meaningful. Instead, one hopes to give conditions on $f $ and $g$ which ensure that $\left( f, g \right) $ is consolable, and which give some properties of $f * g$. 

\begin{eg}
	Consider:
	 \[
		 f\left( t \right)  = \begin{cases}
			 1 & t \in  \left[ -1, 1 \right]  \\
			 0 & \text{otherwise}
		 \end{cases}
	.\] 
	Calculate $f * f$. 
	\begin{align*}
		f \left( t - s \right) &= \begin{cases}
			1 & t - s \in \left[ -1, 1 \right] \\
			0 & \text{ow}
		\end{cases} \\
		&= \begin{cases}
			1 & -s \in \left[ t -1 , t + 1 \right] 
		\end{cases} \\
			\implies f * f\left( t \right) &= \int_{R}^{ } f \left( t - s \right) f \left( s \right) d s  \\
						       &= \int_{-1}^{1} f \left( t - s \right) d s   \\
						       &= \lambda\left( \left[ -1, 1 \right] \cap \left[ t - 1, t + 1 \right]  \right)  \\
						       &= \begin{cases}
							       2 - \left| t \right|  & t \in  \left[ -2, 2 \right] \\
							       0 & \text{otherwise}
						       \end{cases}
	.\end{align*}
	\begin{punch}
		Convolutions "smears support" and "smooths" the functions. 
	\end{punch}
\end{eg}
