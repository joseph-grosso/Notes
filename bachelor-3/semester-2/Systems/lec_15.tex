\lecture{15}{Fri 07 Feb 2020 14:33}{Discrete-time convolution (cont'd)}

\subsection{Convolvable pairs of signals with support in $\Z _{\ge  0}\left( \Delta \right) = \left\{ k \Delta \in  \Z\left( \Delta \right)  \mid  k \ge  0 \right\} $}	

If $f, g \in  l_{loc} \left( \Z_{\ge 0}\left( \Delta \right) ; \F \right) $, then the pair $\left( f, g \right) $ is convolvable, and $f * g \in  l_{loc}\left( \Z_{\ge  0}\left( \Delta \right) ; \F \right) $. There is no fussing in this case of whether $f$ and $g$ are in $L^{1}_{loc}$ as in the continuous-time case.

\subsection{Continuity of discrete-time convolution}

We shall give a list of inequalities that give continuity of the mapping $\left( f, g \right) \longmapsto f * g$ in four cases:
\begin{enumerate}
	\item If $f, g \in  l ^{1}\left( \Z \left( \Delta \right) ; \F \right) $, then 
		\[
		 \|f * g\|_{1}\le  \|f\|_{1}\|g \|_{1}
		.\] 
	\item $f \in l ^{p}\left( \Z\left( \Delta \right) ; \F \right) $, $g \in  l^{2}\left( \Z\left( \Delta \right) ; \F \right) $, $\frac{1}{p} + \frac{1}{q} = 1 + \frac{1}{r}$:
		\[
		\| f * g \|_{r} \le  \| f \|_{p}\| g \|_{q}
		.\] 
		For $f \in  l _{loc}\left( Z_{\ge 0}\left( \Delta \right) ; \F \right) $, define 
		\[
			\| f \|_{N, p} = \left( \sum_{j = 0}^{N} \left| f \left( j \Delta \right)  \right| ^{p} \right) ^{\frac{1}{p}}
		.\] 
		Then we can show that $\left( f _{j} \right) _{j \in  \Z_{\ge 0}}$ converges to zero in the $p$-seminorms for $l _{loc} \left( \Z_{\ge  0}\left( \Delta \right) : \F \right) $ if and only if $\lim_{j \to \infty} \|f _{j}\|_{N, p} = 0 $for every $N \in  \Z_{\ge  0}$.
	\item $f, g \in  l_{loc}\left( \Z_{\ge 0}\left( \Delta \right) ; \F \right) $ then 
		\[
		\| f * g\|_{N, 1}\le  \| f\|_{N, 1} \| g\|_{N, 1}
		.\] 
	\item $f, g \in  l_{loc}\left( \Z_{\ge 0}\left( \Delta \right) ; \F \right) $, $\frac{1}{p} + \frac{1}{q} = 1 + \frac{1}{r}$, 
		\[
		\| f * g\|_{N, r} \le  \| f\|_{N, p}\| g\|_{N, q}
		.\] 
\end{enumerate}

\section{System Theory}

We will talk about eight kinds of systems about eight kinds of systems. These are 
\[
\left\{ \text{systems, linear systems} \right\} \times  \left\{ \text{state-space, input/output} \right\} \times  \left\{ \text{cts-time, disc-time} \right\} 
.\] 
\section{Continuous-time states space systems (CTSSS)}
\begin{definition}
	Let $X \subseteq \real ^{n}$ be open (the state space). Let $U \subseteq \real ^{m}$ (the control-value space), let $\tdomain \subseteq \real$ be continuous time-domain (the time-domain), let $\U \subseteq U^{\tdomain}$ (inputs or controls), let $f : \tdomain \times  X \times  U \to \real^{n}$ (the dynamics), and let  $h  : \tdomain \times  X \times  U  \to \real ^{k}$ (the output map). The sextuple $\left( X, U, \tdomain, \U, f, h \right) $ is a continuous-time state space system. 
\end{definition}

\begin{definition}
	A controlled trajectory of a CTSSS $\sum = \left( X, U, \tdomain, \U, f, h \right)  $ is a pair $\left( \xi , \mu \right) $ where $\mu \in  \U$ is defined on $\tdomain ' \subseteq \tdomain$ and where $\xi : \tdomain '  \to X$ is locally absolutely continuous and satisfies 
	\[
		\hat{\xi}\left( t \right) = f\left( t, \xi\left( t \right) , \mu\left( t \right)  \right) 
	.\] 
\end{definition}

\begin{definition}
	A controlled output for $\sum$ is a pair $\left( \eta, \mu \right) $ where $\mu \in  \U$ is defined on $\tdomain ' \subseteq \tdomain$ and $\eta$ satisfies 
	\[
		\eta\left( t \right) = h \left( t, \xi\left( t \right) , \mu\left( t \right)  \right) 
	.\] 
	where $\left( \xi, \mu \right) $ is a controlled trajectory. Thus a controlled output $ \left( \eta, \mu \right) $ satisfies 
	\begin{align*}
		\dot{\xi} \left( t \right) &= f\left( t, \xi\left( t \right) , \mu\left( t \right)  \right)  \\
		\eta\left( t \right) &= h \left( t, \xi\left( t \right) , \mu\left( t \right)  \right) 
	.\end{align*}
\end{definition}


\begin{question}
	\begin{enumerate}
		\item What properties should the set $\U$ of inputs have?
		\item Given properties of $\U$, what properties should $f $ have so that controlled trajectories exist?
		\item CTSS are examples of general time systems. As general time systems, what properties do they have? 
		\item We will care about continuity of the map $\mu \longmapsto \eta$. 

	\end{enumerate}
\end{question}
