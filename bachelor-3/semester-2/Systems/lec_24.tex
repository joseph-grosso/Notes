\lecture{24}{Fri 06 Mar 2020 15:22}{Lorem Ipsum}

Last time: The LCTSSS. 
\begin{align*}
	\Sigma &= \left( X, U, Y, \tdomain , \U, A, B , C, D \right) \\
		\dot{\eta}\left( t \right) &=  A\left( t \right) \eta\left( t \right)  + B\left( t \right) \mu\left( t \right)  
.\end{align*}

\subsection{Linearization of CTIOS}

This is too complicated, since $g : L^{p} \to L^{q}$, and since both are $\infty$-dimensional, it is it is difficult to talk about derivatives. 

\subsection{Linearization of DTSSS}

Let $\Sigma$ be a DTSSS, $\left( \xi_0, \mu_0 \right) $ be a reference trajectory. Let 
\begin{align*}
	\nu &= \xi  - \xi_0,\\
	\omega &= \mu - \mu_0, \\
	\gamma &= \eta - \eta_0 
.\end{align*}
Therefore, 
\begin{align*}
	\xi_0 \left( t + \Delta	 \right)  + \nu \left( t_0 + \Delta \right) &= f\left( t , \xi_0 + \nu, \mu + \omega \right)  \\
	\eta_0\left( t \right)  + \gamma \left( t \right) &= h\left( t, \xi_0 + \nu , \mu + \omega \right)  
.\end{align*}
The Taylor expansion of $f$ and $h $, keeping first order terms, is 
\begin{align*}
	\nu\left( t + \Delta \right) &= D_1 f\left( t, \xi_0, \mu_0 \right) \nu + D_2 f\left( t, \xi_0 ,\mu_0 \right) \omega \\
	\gamma\left( t  \right) &= D_1 h\left( t, \xi_0, \mu_0 \right) \nu + D_2 h\left( t, \xi_0, \mu_0 \right) \omega 
.\end{align*}
This is a new system $\Sigma_{L\left( \xi_0, \mu_0 \right) }$, $\nu$ state, $\omega$ input, $\gamma$ output.

We can linearize the system (same as CTSSS's): $\Sigma_L$ is a new system with state $\xi$, $\nu$, output $\eta$, $\gamma$, input $\mu$, $\omega$.
