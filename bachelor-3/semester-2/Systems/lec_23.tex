\lecture{23}{Wed 04 Mar 2020 15:28}{Linearisation of CTSSS's (cont'd)}

We note that we can combine the equations for a controlled output $\left( \eta, \mu \right) $ with the equations for linearization about $\left( \eta, \mu \right) $
\begin{align*}
	\dot{\xi} \left( t \right) &= f \left( t, \xi\left( t \right) , \mu\left( t \right)  \right)  \\
	\dot{\nu} \left( t \right)  &= D_1 f\left( t, \xi\left( t \right) , \mu\left( t \right)  \right) \cdot \nu\left( t \right)  + D_2 f\left( t, \xi\left( t \right) , \mu\left( t \right)  \right) \cdot \omega \\
	\eta\left( t \right) &= h\left( t, \xi\left( t \right) , \mu\left( t \right)  \right)  \\
	\gamma\left( t \right)  &=D_1 h\left( t, \xi\left( t \right) , \mu\left( t \right)  \right) \cdot \nu\left( t \right)  + D_2 h\left( t, \xi\left( t \right) , \mu\left( t \right)  \right) \cdot \omega \\
.\end{align*}
The first two equations are known as the state ODE's and the second two equations are known as output equations.

This suggests a linearisation, not about a reference trajectory, but of the whole system. The state is $\left( \xi\left( t \right) ,\nu\left( t \right)  \right) $, the input is $\left( \mu\left( t \right) , \omega \left( t \right)  \right) $, the output is $\left( \eta\left( t \right) , \gamma\left( t \right)  \right) . $

\begin{definition}
	The linearisation of $\Sigma$ is 
	\[
		\Sigma_{L} = \left( X \times  \real^{n}, \mu \times \real^{m}, \tdomain , \U \times L^{\infty}_{loc}\left( \left( \tdomain  \right) ; \real^{m} \right) , f_L , h_l  \right) 
	\] 
	where 
	\begin{align*} 
		f_L: \tdomain \times \left( X \times  \real^{n} \right) \times \left( U \times \real^{m} \right)  &\longrightarrow \real^{n}\times \real^{n} \\
		\left( t, \left( x, \nu \right) , \left( u, \omega \right)  \right)  &\longmapsto f_L(\left( t, \left( x, \nu \right) , \left( u, \omega \right)  \right) ) \\
										     &= \left( f\left( t, x, u \right) , D_1 f\left( t, x , u  \right) \cdot \nu + D_2 f\left( t, x , u  \right) \cdot \omega  \right) \\ 
		h_L: \tdomain \times \left( X \times  \real^{n} \right) \times \left( U \times \real^{m} \right)  &\longrightarrow \real^{k}\times \real^{k} \\
		\left( t, \left( x, \nu \right) , \left( u, \omega \right)  \right)  &\longmapsto h_L(\left( t, \left( x, \nu \right) , \left( u, \omega \right)  \right) ) \\ 
										     &= \left( h\left( t, x, u \right) , D_1 h\left( t, x , u  \right) \cdot \nu + D_2 f\left( t, x , u  \right) \cdot \omega  \right) 
	.\end{align*}
\end{definition}

A controlled trajectory $\left( \left( \xi, \nu \right) , \left( \mu, \omega \right)  \right) $ giving rise to a controlled output $\left( \left( \eta, \gamma \right) , \left( \mu, \omega \right)  \right) $ should be thought of as first giving a controlled trajectory $\left( \xi, \mu \right) \in C_{traj}\left( \Sigma \right) $ which gives the controlled output $\left( \eta, \mu \right) \in C_{out}\left( \Sigma \right) $ and then, second, determining the linearization about $\left( \eta, \mu \right) $ as last time. Thus, $\Sigma_{L}$ contains $\Sigma_{L, \left( \xi, \mu \right) }$.

\subsection{Linearisation of CTSSS's about equilibria}

\begin{definition}
	A controlled equilibrium for $\Sigma = \left( X, U, \tdomain , \U, f, h \right) $ is a pair $\left( x_0, u_0 \right) \in X \times  U$ such that $f\left( t, x_1, u_0 \right) = 0$ for all $t \in  \tdomain $. 
\end{definition}

\begin{note}
	$\left( x_0, u_0 \right) $ is a controlled equilibrium if and only if $\left( \xi, \mu \right) $ wt $\xi\left( t \right) = x_0$ and $\mu\left( t \right) = u_0$ is a controlled trajectory. 
	\begin{proof}
		Suppose that $\left( x_0, u_0 \right) $ is a controlled equilibrium. Then $f\left( t, x_0, u_0 \right) = 0$. Now consider $\left( \xi, \mu \right) $ as above. Then $\dot{\xi} \left( t \right) = 0$. Then 
		\begin{align*}
			\dot{\xi}\left( t \right) = 0 &= f\left( t, x_0, u_0 \right) = f\left( t, \xi\left( t \right) , \mu\left( t \right)  \right) \\
			\implies \left( \xi, \mu \right) \in C_{traj}\left( \Sigma \right) 
		.\end{align*}
		Now, with $\left( \xi, \mu \right) $ as above, note that 
		 \[
			 0 = \dot{\xi}\left( t \right) = f\left( t, \xi\left( t \right) , \mu\left( t \right)  \right) = f\left( t, x_0, u_0 \right) 
		.\] 
	\end{proof}
	Thus, since $t \longmapsto \left( x_0, u_0 \right) $ is a controlled trajectory, we can linearise about it.
\end{note}

\begin{definition}
	Let $\left( x_0, u_0 \right) $ be a controlled equilibrium for $\Sigma = \left( X, U, \tdomain , \U, f , h  \right) $. The linearisation of $\Sigma$ about $\left( x_0, u_0 \right) $ is 
	\begin{align*}
		\Sigma _{L, \left( x_0, u_0 \right) } &= \left( \real^{n}, \real^{m}, \tdomain , L^{\infty}_{loc}\left( \tdomain ; \real^{m} \right) , f_{L, \left( x_0, u_0 \right) }, h_{L, \left( x_0, u_0 \right) } \right)  
	.\end{align*}
	with
	\begin{align*}
		f_{L, \left( x_0, u_0 \right) }\left( t, \nu, \omega \right) &= D_1 f\left( t, x_0, u_0 \right) \cdot \nu + D_2 f\left( t, x_0, u_0 \right)  \cdot \omega \\
		h_{L, \left( x_0, u_0 \right) }\left( t, \nu, \omega \right) &= D_1 h\left( t, x_0, u_0 \right) \cdot \nu + D_2 h\left( t, x_0, u_0 \right)  \cdot \omega
	.\end{align*}
	For autonomous systems, we can define 
	\begin{align*}
		A_{\left( x_0, u_0 \right) } &= D_1 f\left( x_0, u_0 \right)  \\
		B_{\left( x_0, u_0 \right) } &= D_2 f\left( x_0, u_0 \right)  \\
		C_{\left( x_0, u_0 \right) } &= D_1 h\left( x_0, u_0 \right)  \\
		l_{\left( x_0, u_0 \right) } &= D_2 h\left( x_0, u_0 \right) 
	.\end{align*}
	Then controlled outputs for the linearisation satisfy
	\begin{align*}
		\dot{\nu} \left( t \right) &= A_{\left( x_0, u_0 \right) }\nu \left( t \right) + B_{\left( x_0, u_0 \right) }\omega \left( t \right) \\
		\gamma\left( t \right) &= C_{\left( x_0, u_0 \right) }\nu\left( t \right) + D_{\left( x_0, u_0 \right) }\omega\left( t \right) 
	.\end{align*}
	This is an LTI system.
	
\end{definition}
