\lecture{6}{Thu 16 Jan 2020 09:35}{}

Limits of integration when doing multiple integrals. Say $\left( x_1, \ldots, x_{n} \right) : \R^{n} \to \R $ is a real-valued function on $\R^{n}$ (joint pdf, non-negative, $A \subseteq \R^{n}$. 

We want 
\[
	\int \ldots \int_{A} f\left( x_1, \ldots, x_{n} \right) dx_1 \ldots dx_{n}
.\] 
\begin{figure}[ht]
    \centering
    \incfig{xonextwograph}
    \caption{Graph of an example of the probability distributions of $x_1$ and $x_2$}
    \label{fig:xonextwograph}
\end{figure}


In general the limits of integration for $x_1$ will depend on $x_2$. 

\subsection{General Approach}

\begin{enumerate}
	\item Pick an order in which to do the 1-dimensional integrals wisely (i.e., try to make the integrations simpler as opposed to more complicated. 
	\item Figure out the limits of integration: 
		\begin{enumerate}
			\item start with the outer integral - vary $(x_1, \ldots, x_{n})$ over A, then all the possible values of $x_{n}$ give the limits for the outer integral. 
			\item Next, consider the limits of the second outermost (i.e., $x_{n-1}$ integral. To do this, for each fixed $x_{n}$ you vary $\left( x_1, \ldots, x_{n-1} \right) $ over $A$ (with $x_{n}$ fixed) and note all the possible values of $x_{n-1}$. This will give the limits for $x_{n-1}$ (which may depend on $x_{n}$).
			\item Continue in this way. For variable $x_{j}$ you fix $x_{j+1}, \ldots, x_{n}$, let $x_1, \ldots, x_{j}$ vary over $A$. This gives limits for $x_{j}$ (which may depend on $x_{j + 1}, \ldots, x_{n}$). 
		\end{enumerate}
\end{enumerate}

\begin{example}
	Suppose $\left( X_1, X_2, X_3 \right) $ are jointly continuous and have joint pdf 
	\[
		f_{X}\left( x_1, x_2, x_3 \right)  = \begin{cases}
			1 & \text{for } 0 \le x_1, x_2, x_3 \le 1\\
			0 & \text{ otherwise}
		\end{cases}
	.\] 

	What is the probability that the quadratic equation $X_1 y^{2} + X_2 y + X_3 $ has real roots?

	Let $A = \left\{ \left( x_1, x_2, x_3 \right) \in  \left[ 0,1 \right] ^3 \mid x_1 y^{2} + x_2 y + x_3 \text{  has real roots} \right\}$. The quadratic equation $x_1 y^2 + x_2 y + x_3$ has real roots if and only if $x_2^2 - 4x_1 x_3 \ge 0$. 

	So $A = \left\{ \left( x_1, x_2, x_3 \right) \in  \left[ 0,1 \right] ^3 \mid x_2 ^2 - 4 x_1 x_3 \ge 0 \right\}$. Then:
	\begin{align*}
		P\left( A  \right) &= \int\int\int_{A} \left( 1 \right) dx_2 dx_1 dx_3  \\
				   &= \int_{0}^{1} \int_{x_1}\int_{x_2} \left( 1 \right) dx_2 dx_1 dx_3 \\
				   &= \int_{0}^{\frac{1}{4}} \int_{0}^{1} \int_{\sqrt{4 x_1 x_3} }^{1} \left( 1 \right) dx_2 dx_1 dx_3     \\
				   &+ \int_{\frac{1}{4}}^{1} \int_{0}^{\frac{1}{4x_3}} \int_{\sqrt{4 x_1 x_3} }^{1} \left( 1 \right) dx_2 dx_1 dx_3 \\
				   &= \text{Solve this on your own time} 
	.\end{align*}
\end{example}
